\chapter{Experimentelle Untersuchung der möglichen Strategien}





\section{Experimentales Setup}

Welche Hardware und damit zusammenhängend welche Versionen der dazugehörenden Software ind vorhanden ---- Daraus erwächst die Auswahl welche Strategien überhaupt möglich sind 


\section{Überblick über die möglichen Strategien}

Welchen Strategien aus Kapitel 2 sind überhaupt durchführbar und welche sind kombinierbar?
Hier werden nur die Strategien aufgeführt, welche überhaupt auf vernünftig grossen Datensätzen funktionieren und von der Technik her möglich sind.
Die Strategien sind aufgeteilt in Unterkapitel. 

Alle möglichen Kombinationen von Strategien sind zuviele. Daher sinnvolle Vorauswahl treffen.  

\subsection{Zahlenformate}

\begin{itemize}
 \item FP16 bereits probiert
 \item DFP 16 without Swalp
 \item DFP 16 with Swalp
\end{itemize}



\subsection{Beschleunigung der Berechnung des Gradientenabstiegverfahren}



\subsection{Verfahren um weniger Trainingsdaten zu verwenden}


\subsection{Strukturelle Veränderungen}


\subsection{andere Herangehensweisen}




\subsection{Tensorflow vs. PyTorch}



\section{Evaluation der Ergebenisse}
