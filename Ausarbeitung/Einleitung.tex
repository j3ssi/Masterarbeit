\chapter{Einleitung}
\label{sec:EinleitungGesamt}

\section{Motivation und Hintergrund dieser Arbeit}






Trainingszeiten für Convolutional Neural Networks (CNNs) wachsen schnell mit der Komplexität des Datensatzes und der Größe des Netzes. Neben der Trainingszeit eines bestimmten Netzes kostet vorallem die eventuell nötige Anpassung der Hyperparameter des Netzes und weitere Trainingsdurchläufe Zeit. Ziel dieser Masterarbeit ist zu untersuchen in wie fern dieser Prozess beschleunigt und automatisiert werden kann. Um den Prozess des Findens der besten Hyperparameter/ Netzarchitektur zu automatisieren gibt es bereits einige Arbeiten. Viele dieser Herangehensweisen lassen sich allerdings nicht direkt auf einen unbekannten Datensatz anwenden oder der Zeitaufwand für diese Verfahren ist enorm. Ein Verfahren, dass diese Nachteile nicht hat ist das Ressourcen beschränktes Strukturlernen tiefer Netzwerke. Diese Verfahren wird in dieser Arbeit mit einer Kombination aus zwei anderen Verfahren verglichen. Diese Kombination besteht aus einem Verfahren, welches das Netz währenddem Training beschneidet und einem Verfahren, welches das Netz breiter oder tiefer machen kann.

\section{Ziel der Arbeit}




\section{Ergebnisse der Arbeit}


\section{Aufbau der Arbeit}
In Kapitel \ref{sec:wissenschaft} wird zunächst der aktuelle Stand der Wissenschaft erläutert. Zu diesem Zweck werden zunächst in Unterkapitel \ref{sec:conv} die Grundlagen und Funktionsweisen eines CNNs erklärt. Die in dieser Arbeit verwendete CNN-Architektur ResNet wird darauf aufbauend in Unterkapitel \ref{sec:res} beschrieben.


Unterkapitel Suche Literatur \ref{sec:suche}

PruneTrain \ref{sec:prunetrain}

Net2Net \ref{sec:net2net}

MorphNet \ref{sec:morphnet}

Automatische Architektursuche \ref{sec:auto}

Zeitsparen: \ref{sec:time}

Additive Methoden \ref{sec:add}

Experimente:\ref{sec:experimente}

Setup der Experimente \ref{sec:setup}

PruneTrain Experimente: \ref{sec:ptexperimente}

Net2Net Experimente: \ref{sec:net2netexperimente}

MorphNet: \ref{sec:morphexperimente}

Net2Net + PruneTrain: \ref{sec:ptpnet2net}

Evaluation: \ref{sec:evaluation}

Fazit: \ref{sec:fazit}
\color{black}
