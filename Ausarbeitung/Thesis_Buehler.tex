% ----------------------------------------------------------------------
%
%                         Computer Science VII
%
%                   http://ls7-www.cs.uni-dortmund.de
%
%   questions and representations: info@ls7.cs.uni-dortmund.de
%
%   status: 20.12.2017
%
% ----------------------------------------------------------------------

\RequirePackage{ifthen}
\newcommand \Thesistyp{Masterarbeit}
\newcommand \Author{Jessica Buehler}
\newcommand \Thesistitle{Zeit-effizientes Training von Convolutional Neural Networks}
\newcommand \FirstSupervisor{Prof.~Dr.~Heinrich~M\"uller}
\newcommand \SecondSupervisor{M.Sc.~Matthias~Fey}
\newcommand \FirstChair{Lehrstuhl VII}
\newcommand \FirstChairTitle{Informatik}


% -----------------------------------------------------------------------------------------
% Option: Zweiter Lehrstuhl
\newboolean{boolNoSecondChair}
\setboolean{boolNoSecondChair}{true} % boolNoSecondChair==false: second chair involved
\ifthenelse{\boolean{boolNoSecondChair}}{
\newcommand \SecondChair{}
\newcommand \SecondChairTitle{}
}{
\newcommand \SecondChair{Computer Science XII}
\newcommand \SecondChairTitle{Embedded Systems}
}

\RequirePackage{ifpdf} \ifpdf
  \pdfoutput=1
  \pdftrue
  \message{pdfLaTeX}
  
  \documentclass[pdftex,12pt,a4paper,twoside,numbers=noenddot]{scrbook}
  \usepackage{float}
  %\usepackage[pdftex]{thumbpdf}
  \usepackage[pdftex]{pdflscape}
  \usepackage[pdftex]{graphicx}
  \usepackage[pdftex, pdfencoding=auto]{hyperref}
  \usepackage{pdfpages}
  \pdfoutput=1
  \pdfcompresslevel=9
  \DeclareGraphicsExtensions{.pdf,.jpg,.png}
\else
  \pdffalse
  \message{LaTeX}
  \documentclass[dvips,12pt,a4paper,twoside,numbers=noenddot]{scrbook}
  \usepackage{float}
  \usepackage{graphicx}
  \usepackage{epsf}
  \usepackage[dvips, pdfencoding=auto]{hyperref}
  \DeclareGraphicsExtensions{.eps}
\fi

\usepackage{silence}
\WarningFilter{scrbook}{Usage of package `fancyhdr'}
%
\hypersetup
{
    pdfauthor = {\Author},
    pdftitle = {\Thesistitle},
    pdfsubject = {\Thesistyp, TU Dortmund, Fakultaet Informatik},
    pdfproducer = {LaTeX},
    pdfview = FitV,
    pdfstartview = FitV,
    pdfhighlight = /I,
    pdfborder = 0 0 0,
    colorlinks = false,
    bookmarksopen,
    bookmarksopenlevel = 1,
    bookmarksnumbered = false,
    plainpages = false
}%


%
\usepackage[a4paper,left=3.5cm,right=2.5cm,bottom=3.5cm,top=3cm]{geometry}
\setlength{\headheight}{15pt}
% -------------------------------------------------------------------
%
\usepackage{amsmath,amssymb}
%\usepackage{flafter}
\usepackage{subfigure}

% -------------------------------------------------------------------
\usepackage{ifthen}
\usepackage[colorinlistoftodos,prependcaption]{todonotes}

% -------------------------------------------------------------------
\usepackage[absolute,overlay]{textpos}
\setlength{\TPHorizModule}{1mm}
\setlength{\TPVertModule}{\TPHorizModule}
\textblockorigin{0mm}{0mm}
\usepackage{fix-cm}
\usepackage{setspace}
\usepackage{scrhack}
% -------------------------------------------------------------------
%
\usepackage[german]{babel}
\usepackage[utf8]{inputenc}
\usepackage[T1]{fontenc}
\usepackage{ae,aecompl}

% -------------------------------------------------------------------
\usepackage[numbers,sort,square]{natbib}

% -------------------------------------------------------------------
\usepackage[babel,german=quotes]{csquotes}

% -------------------------------------------------------------------
\usepackage{url}
%\usepackage[hyphenbreaks]{breakurl}
%\def\UrlBreaks{\do\a\do\b\do\c\do\d\do\e\do\f\do\g\do\h\do\i\do\j\do\k\do\l%
%\do\m\do\n\do\o\do\p\do\q\do\r\do\s\do\t\do\u\do\v\do\w\do\x\do\y\do\z\do\0%
%\do\1\do\2\do\3\do\4\do\5\do\6\do\7\do\8\do\9\do\-}%

% -------------------------------------------------------------------
\usepackage[margin=0pt,font=small,labelfont=bf]{caption}

% -------------------------------------------------------------------
\usepackage{booktabs}

% -------------------------------------------------------------------
\usepackage{eurosym}

% -------------------------------------------------------------------
\renewcommand{\baselinestretch}{1.25}
\renewcommand{\topfraction}{0.9}
\renewcommand{\bottomfraction}{0.8}

% -------------------------------------------------------------------
%\clubpenalty = 10000
%\widowpenalty = 10000 \displaywidowpenalty = 10000

\parindent=0cm


% -------------------------------------------------------------------
\usepackage{fancyhdr}
\usepackage{extramarks}

\pagestyle{fancy}
\renewcommand{\chaptermark}[1]{\markboth{#1}{}}
\renewcommand{\sectionmark}[1]{\markright{#1}{}}

\fancyhf{}
\fancyhead[LE,RO]{\thepage}
\fancyhead[RE]{\textit{\nouppercase{\leftmark}}}
\fancyhead[LO]{\textit{\nouppercase{\rightmark}}}

\fancypagestyle{plain}{ %
\fancyhf{} % remove everything
\renewcommand{\headrulewidth}{0pt} % remove lines as well
\renewcommand{\footrulewidth}{0pt}} \pagestyle{headings}



% -------------------------------------------------------------------
\usepackage{color}
\definecolor{TUGreen}{rgb}{0.517,0.721,0.094}
\definecolor{TUOrange}{rgb}{1.0,0.7176,0.0}
\definecolor{BrightGray}{gray}{0.9}
\definecolor{DarkGray}{gray}{0.2}
\definecolor{white}{rgb}{1,1,1}
\definecolor{black}{rgb}{0,0,0}
\definecolor{red}{rgb}{1,0,0}




% -------------------------------------------------------------------
\usepackage{listings}

\lstdefinestyle{C++}
{
language=C++,
backgroundcolor=\color{BrightGray},
keywordstyle=\texttt\bfseries,  %\color{TUGreen}\bfseries,
commentstyle=\color{DarkGray},
stringstyle=\color{red},
showstringspaces=false,
basicstyle=\small\color{black},
numbers=left,
captionpos=b,
tabsize=4,
breaklines=true
}


% -------------------------------------------------------------------
% Algorithmen
\usepackage[plain,chapter]{algorithm}
\usepackage{algorithmic}

\usepackage{enumerate}

% -------------------------------------------------------------------
% Algorithmen anpassen
\renewcommand{\algorithmicrequire}{\textit{Eingabe:}}
\renewcommand{\algorithmicensure}{\textit{Ausgabe:}}
\floatname{algorithm}{Algorithmus}
\renewcommand{\listalgorithmname}{Algorithmenverzeichnis}
\renewcommand{\algorithmiccomment}[1]{\color{grau}{// #1}}

\usepackage{verbatim}

% -------------------------------------------------------------------
% -------------------------------------------------------------------
% -------------------------------------------------------------------
\begin{document}


% Front Page ---------------------------------------------------------
%
%\pdfbookmark[0]{Titlepage}{title}

%\pdfbookmark{Deckblatt}{pdf:title}
\pagenumbering{alpha}
\pagestyle{empty}
%========================================================================================
% TU Dortmund, Computer Science VII
%========================================================================================

\begin{titlepage}

\begin{textblock}{150}(30.5,10.75)%
\raggedright
\includegraphics[width=83.25mm]{images/tud_logo_cmyk.pdf}%
\end{textblock}

\begin{textblock}{150}(21.2,41.6)%
\raggedright\sffamily%\Huge
{\color{red}\rule{5mm}{5mm}}
\end{textblock}

\begin{textblock}{150}(30.4,38)%
\raggedright
\includegraphics[height=13mm]{images/fi_text.pdf}
\end{textblock}

\begin{textblock}{89}(35.0,62.75)%
\begin{minipage}{80mm}
	\vfill
	\begin{center}
	\fontsize{24pt}{24pt} \sffamily
	\Thesistyp
	
	\vspace{1cm}
	\begin{onehalfspace}
    \fontsize{18pt}{18pt}
    \sffamily \Thesistitle
    \end{onehalfspace}
	
	\vspace{12mm}
\begin{onehalfspace}
	{\fontsize{14pt}{14pt}\sffamily \Author

	\today}
 \end{onehalfspace}
	\end{center}
	\vfill
\end{minipage}\end{textblock}

\begin{textblock}{150}(44.25,208)%
\begin{minipage}{120mm}
	\large
	\raggedright
	\sffamily
    {\fontsize{14pt}{14pt}
	\textbf{Supervisors:}\\
	\FirstSupervisor\\
	\SecondSupervisor\\}
\end{minipage}
\end{textblock}



\begin{textblock}{150}(44.25,242.0)%
\ifthenelse{\boolean{boolNoSecondChair}}{
%
\begin{minipage}{120mm}
	\fontsize{11.75pt}{11.75pt}\selectfont
	\raggedright
	\sffamily
	\textcolor{TUGreen}{\FirstChair}\\
	\textcolor{TUGreen}{\FirstChairTitle}\\
    \textcolor{TUGreen}{TU Dortmund}
\end{minipage}
%
}{
%
\begin{tabular*}{\textwidth}[t]{c c}%
  \begin{minipage}[t]{70mm}
    \raggedright
	\sffamily
    \textcolor{TUGreen}{\SecondChair}\\
    \textcolor{TUGreen}{(\SecondChairTitle)}\\
    \textcolor{TUGreen}{TU Dortmund}
    \end{minipage}
    \hspace*{0.5cm}
    \begin{minipage}[t]{70mm}
    \raggedright
	\sffamily
    \textcolor{TUOrange}{\SecondChair}\\
    \textcolor{TUOrange}{(\SecondChairTitle)}\\
    \textcolor{TUOrange}{TU Dortmund}
  \end{minipage}
\end{tabular*}
%
}
%
\end{textblock}



\vspace*{20cm}



\end{titlepage}

\pagestyle{empty} \cleardoublepage

% Inhaltsverzeichnis -------------------------------------------------
%
%\pdfbookmark{Inhaltsverzeichnis}{pdf:toc}
\tableofcontents

\cleardoublepage \pagestyle{headings}

% Mathematische Notation -----------------------------------------------
%
\pagestyle{empty}

\listoftodos

%\pdfbookmark{Mathematische Notation}{pdf:Notation}
%========================================================================================
% TU Dortmund, Computer Science VII
%========================================================================================

\chapter*{Mathematische Notation} \label{Notation}

\newcommand{\tabdummy}{\midrule[0pt]}

\begin{tabular}{p{0.25\textwidth}p{0.65\textwidth}}
  \textbf{Notation} & \textbf{Bedeutung} \\ \toprule[1pt]
   $\mathbf{x_{i}}$ & Eingabe in das Netz \\ \tabdummy
   $\mathfrak{B}$ & Menge der Batches des Datensatzes $\mathfrak{B}=\left\{ \mathcal{B}_{1} , \ldots \mathcal{B}_{k}   \right\} $\\ \tabdummy
   $\mathcal{B}$ & ein Batch mit $m$ Elemente $\mathcal{B}=\left\{ \mathbf{x}_1, \ldots, \mathbf{x}_m \right\} $\\ \tabdummy
   $\mathbf{u}_{i,j}$ & Ausgabe der Faltungsschicht j wobei in das Netz $\mathbf{x}_i$ eingegeben wurde.\\ \tabdummy
   $\mathbf{v}_{i,j}$ & Ausgabe der Batchnormalisierungsschicht j wobei in das Netz $\mathbf{x}_i$ eingegeben wurde.\\ \tabdummy
   $\mathcal{W}$ & Gewichte der Schichten $\mathcal{W}= \left\{ W_1, \ldots, W_G \right\} $ geordnet nach Stelle der Schicht im Vorwärtsdurchgang. \\ \tabdummy
   $\mathcal{W}^k$ & Gewichte des k-ten Trainingsdurchlaufs  \\ \tabdummy
   $f\left( \mathbf{x_i}, \mathcal{W}\right) $ & Funktion, die das CNN berechnet. $f\left( \mathbf{x}_i, \mathcal{W}\right)$ berechnet eine Klassenzugehörigkeit für $\mathbf{x}_i$ \\ \tabdummy
   $y_i$ & tatsächliche Klasse von $\mathbf{x}_i$ \\ \tabdummy     
   $l\left(f( \mathbf{x_i}, \mathcal{W}\right), y_i) $ & Funktion, die das CNN berechnet  
\end{tabular}

\cleardoublepage

\pagenumbering{arabic}
\pagestyle{fancy}

% Inhalte --------------------------------------------------------------
%
\chapter{Überblick über die Arbeit}\label{sec:experimente}
Im zweiten Teil dieser Arbeit werden die in Kapitel \ref{sec:wissenschaft} theoretisch betrachteten Methoden praktisch auf einer GPU ausgeführt und evaluiert. Der Überblick über das experimentelle Setup wird in Kapitel \ref{sec:setup} vorgestellt. In Kapitel \ref{sec:konzept} wird das Konzept des praktischen Teils der Arbeit erläutert.

\section{Experimentelles Setup}\label{sec:setup}
\subsection{Hardware}
Die Hardware umfasst einen Server mit 4 GPUs.
Von diesen 4 GPUs haben 2 GPUs jeweils den gleichen Typ:
\begin{itemize}
 \item Geforce GTX 1080 Ti
 \item Geforce RTX 2080 Ti 
\end{itemize}

Beide GPU-Typen arbeiten mit der CUDA Version 10.1. 

Während der Vorbereitung auf diese Experimente hat sich gezeigt, dass Experimente mit einer Geforce GTX 1080 Ti mit den Experimenten der Geforce RTX 2080 Ti nicht vergleichbar sind. Weiterhin lässt sich durch das Verwenden von gemischt präzisen Zahlen nur auf der Geforce RTX 2080 ein Geschwindigkeitsvorteil beim Training feststellen. Aus diesen zwei Gründen wurden alle Experimente auf der Geforce RTX 2080 Ti ausgeführt.  


\subsection{Wahl des Frameworks}

Es wird mit pytorch gearbeitet, da pytorch gegenüber anderen Frameworks eine grössere Flexibiltät erlaubt. Ausserdem ist eine fast vollständige Implementierung von PruneTrain in Pytorch geschrieben. Diese wird im nächsten Kapitel untersucht und soweit erweitert, dass es dem Stand im PruneTrain Paper entspricht.

Pytorch bietet mit cudnn und cuda im Hintergrund gute Möglichkeiten die Trainingszeiten einzelner Epochen zu messen und sie so mit einander zu vergleichen.


\subsection{verwendete Netzarchitektur}\label{sec:archi}
Die PruneTrain Implementierung hat initial mehrere verschiedene Netzarchitekturen zur Auswahl:
\begin{itemize}
 \item AlexNet
 \item ResNet 32/50
 \item vgg 8/11/13/16
 \item mobilenet
\end{itemize}

Diese Auswahl an Netzarchitekturen ist zu umfangreich, um alle diese Architekturen auf den vorgestellten Methoden zu evaluieren. Daher wird im Rahmen dieser Arbeit nur auf ResNet gearbeitet. Diese Entscheidung liegt daran, dass Resnets durch ihre Kurzschlussverbindungen gut mit sehr tiefen Netzstrukturen unmgehen können, ohne grosses Klassifikationsleistungsverluste dank Overfitting. Dies ist vorallem wichtig, wenn das Netz mit Hilfe des Operator für tieferes Netz noch tiefer gemacht werden soll. Die ResNet Struktur wird in der Implementierung so verändert, dass angeben werden kann wie tief das Netz sein soll. Das ResNet wird hier nicht mehr nur mit einer Zahl identifiziert sondern es wird angegeben, wieviele
\begin{itemize}
 \item $s$: Anzahl an Phasen, die das ResNet hat
 \item $N=[n_1, \ldots, n_S]:$ Anzahl von Blöcken pro Phase 
 \item $l$: Anzahl von (Conv+Batch)-Layer pro Block
 \item $[k_1, \ldots,k_S ]:$ Breite der Schichten je Phase
 \item $b$: Boolean Parameter, der angibt ob die Blöcke im Netz die Bootleneck-Eigenschaft haben
\end{itemize}
das jeweilige ResNet hat. Diese Vorgehensweise hat den Vortei, dass für ein im Verlauf tieferes beziehungweise breitere Netz eine Vergleichsmöglichkeit besteht. Dies bedeutet, dass das Netz welches im Verlauf entsteht auch direkt erstellt werden kann.


\subsection{Baseline Netz}\label{sec:baseline}

Um die Ergebnisse der Experimente in den folgenden Kapiteln einschätzen zu können wird ein ResNet, ohne Anpassungen um Trainingszeit zu sparen, trainiert. In Tabelle \ref{tab:baseline} ist die Struktur dieses Netze zu sehen. Das breite Baseline-Netz wird dabei für die Evaluierung des Beschneiden des Netzes verwendet. Das schmallere Baseline-Netz wird für die Evaluierung der Methoden, die das Netz breiter machen verwendet.

Das Netz hat drei Phasen $(s=3)$, wobei jeder der Phasen 5 Blöcke hat $(N=[5,5,5])$. Pro Basisblock sind zwei (Conv+Batch)-Schichten vorhanden $(l=2)$. Bei einem Übergangsblock, der als erster Block in einer neuen Phase bei einer Vergrösserung der Bereit beim Phasenübergang genutzt wird ist eine (Conv+Batch)-Schicht mehr vorhanden. Eine grafische Darstellung der Blöcke ist in Abbildung \ref{abb:blocks} zu sehen.




\begin{figure}[]
   \begin{minipage}[b]{.4\linewidth} % [b] => Ausrichtung an \caption
      \includegraphics[width=0.8\linewidth]{KapitelPartB/Images/Basisblock.png}
      \caption{Basisblock}
   \end{minipage}
   \hspace{.1\linewidth}% Abstand zwischen Bilder
   \begin{minipage}[b]{.4\linewidth} % [b] => Ausrichtung an \caption
      \includegraphics[width=0.8\linewidth]{KapitelPartB/Images/Ubergangsblock.png}
      \caption{Übergangsblock}
   \end{minipage}
   \caption{Grafische Darstellung Basis- und Übergangsblock}
   \label{abb:blocks}
\end{figure}

\begin{table}[]
\begin{tabular}{|l|l|l|l|l|l|}
\hline
      &                & \multicolumn{2}{c|}{breites Baseline-Netz} &\multicolumn{2}{c|}{schmalles Baseline-Netz} \\ 
Phase & Schicht/Block  & \#Eingangs- & \#Ausgangs-       & \#Eingangs- & \#Ausgangs-    \\
      &                & \multicolumn{2}{c|}{kanäle}     & \multicolumn{2}{c|}{kanäle}  \\ \hline
      & Conv 1 + Bn 1  & 3                & 16           & 3           & 8              \\ \hline \hline
1     & Basisblock     & 16               & 16           & 8           & 8              \\ \hline
      & Basisblock     & 16               & 16           & 8           & 8              \\ \hline
      & Basisblock     & 16               & 16           & 8           & 8              \\ \hline
      & Basisblock     & 16               & 16           & 8           & 8              \\ \hline
      & Basisblock     & 16               & 16           & 8           & 8              \\ \hline \hline
2     & Übergangsblock & 16               & 32           & 8           & 16             \\ \hline
      & Basisblock     & 32               & 32           & 16          & 16             \\ \hline
      & Basisblock     & 32               & 32           & 16          & 16             \\ \hline
      & BasisBlock     & 32               & 32           & 16          & 16             \\ \hline
      & BasisBlock     & 32               & 32           & 16          & 16             \\ \hline \hline
3     & Übergangsblock & 32               & 64           & 16          & 32             \\ \hline
      & Basisblock     & 64               & 64           & 32          & 32             \\ \hline
      & Basisblock     & 64               & 64           & 32          & 32             \\ \hline
      & Basisblock     & 64               & 64           & 32          & 32             \\ \hline
      & Basisblock     & 64               & 64           & 32          & 32             \\ \hline \hline
      & Linear         & 64               & 10           & 32          & 10             \\ \hline
\end{tabular}
\caption{Struktur des Netzes}
\label{tab:baseline}
\end{table}


\subsubsection{Evaluierung des breiteren Baseline-Netzes}
\color{vermi}
Das Training wird über 180 Epochen durchgeführt. Es werden 5 Experimente durchgeführt. Dabei ergibt sich der in Abbildung \ref{abb:baseAcc1} gezeigte Verlauf der Validierungs-Accuracy für Experiment vier. Bei diesem Training wurde über die gesamten 180 Epochen mit einer Lernrate von 0.1 trainiert. Mit einem Ergebnis von durchschnittlich 81.11 \% über fünf Experimente ist diese Ergebnis leider nicht zufriedenstellend. 

Eine Verkleinerung der Lernrate kann dieses Ergebnis verbessern \cite{CNNBook}. In Abbildung \ref{abb:baseAcc2} wird dargestellt, wie sich der Verlauf ändert durch eine Anpassung der Lernrate bei Epoche 93 und 150. In diesen beiden Epochen wird die Lernrate jeweils auf ein Zehntel verkleinert. 

In Abbildung \ref{abb:baseAcc3} ist ein Boxplot dargestellt, der die Accuracy von jeweils fünf Experimenten mit oder ohne Anpassung der Lernrate vergleicht. Es ergibt sich eine deutliche Verbesserung der Accuracy des Baseline-Netzes von 81.24 \% auf 91.89 \% durch diese Anpassung. Aufgrund dieser Verbesserung werden die Experimente mit Anpassung der Lernrate als Grundlage zum Vergleich mit den Experimenten in den weiteren Kapiteln herangezogen. Sie werden in diesen Kapitel nur mit Baseline-Netz bezeichnet.

\begin{figure}
     \centering
     \subfloat[][]{\includegraphics[width=.49\textwidth]{KapitelPartB/Images/BaseAcc1.png}\label{abb:baseAcc1}}
     \subfloat[][]{\includegraphics[width=.49\textwidth]{KapitelPartB/Images/BaseAcc.png}\label{abb:baseAcc2}}\\
     \subfloat[][]{\includegraphics[width=.49\textwidth]{KapitelPartB/Images/BaseAcc3.png}\label{abb:baseAcc3}}
     \caption{Vergleich zwischen (a) Baseline-Netz ohne Anpassung der Lernrate und (b) Baseline-Netz mit Anpassung der Lernrate in Epoche 93 und 150. (c)  Boxplot der Accuracys}
     \label{abb:BaseAcc}
\end{figure}

Ein weiteres Vergleichskriterium neben der Accuracy sind die durchschnittlichen Trainingszeiten pro Epoche. Für jedes Experiment wird die durchschnittliche Dauer einer Trainingsepoche mittel des arithmetiscen Mittels berechnet.
Die Durchschnittswerte über alle 180 Epoche sind für die zehn Baseline Experimente sind in Tabelle \ref{tab:baselineTime} aufgelistet. Die Durchschnittswerte liegen sehr nah beeinander, der Unterschied zwischen dem grössten und dem kleinsten Durchschnittswert liegt bei 0,28. Damit ergeben sich zwischen den zehn Experimenten keine signifikante Unterschiede. Es wird daher mit Experiment vier eines der Experimente mit Anpassung der Lernrate ausgesucht um für die folgenden Experimente/ Kapitel als Vergleich zu dienen.
\begin{table}[h]
\begin{tabular}{|l|l|l|l|l|l|l|l|l|l|l|} \hline
           & \multicolumn{5}{c|}{Experimente ohne Anpassung}&\multicolumn{5}{c|}{Experimente mit Anpassung} \\
           &\multicolumn{5}{c|}{der Lernrate} &\multicolumn{5}{c|}{der Lernrate}\\
           & 1       & 2      & 3      & 4       & 5       & 1      & 2     & 3      & 4     & 5  \\ \hline 
$\mu$      & 19,49   & 19,53  & 19,50  & 19,48   & 19,84   & 19,57  & 19,56 & 19,53  & 19.53 & 19.62 \\ \hline
\end{tabular}
\caption{Tabelle für Durchschnittswerte und Standardabweichungen der Trainingszeiten der Experimente}
\label{tab:baselineTime}
\end{table}
\color{black}
\subsubsection{Evaluierung des schmalleren Baseline-Netzes}
\color{blue1}
Das schmallere Baseline-Netz wird verwendet, um für MorphNet und Net2Net eine schmalle Variante zu haben. Der Grund hierfür ist, dass bei einer durchschnittlichen Accuracy von 93,19 \% des breiten Baseline-Netzes nicht mehr viel Raum für Verbesserungen bleibt. In Abbildung \ref{abb:baseAcc3} sind die Experimente für das breite und schmalle Baseline-Netz mit der Anpassung der Lernrate gegenüber gestellt. Der Unterschied vom breiten zum schmallen Netz ist ein Accuracy-Verlust von 3,1 \%.    


In Abbildung \ref{abb:baseAccS1} ist abgebildet, wie sich die Accuracy für das schmallere Netz verhält, bei der gleichen Anpassung der Lernrate wie beim breiteren Baseline-Netzen. Der Unterschied in der Accuracy zwischen dem schmallen und breiten Baseline-Netz ist in Abbildung \ref{abb:baseAccS2} abgebildet.


\begin{figure}
     \centering
     \subfloat[][]{ \includegraphics[width=0.5\textwidth]{KapitelPartB/Images/baseAccS.png}\label{abb:baseAccS1}}
     \subfloat[][]{\includegraphics[width=.49\textwidth]{KapitelPartB/Images/baseAccSHisto.png}\label{abb:baseAccS2}}
     \caption{}
     \label{abb:BaseAccS}
\end{figure}


\color{black}

\section{Konzept}\label{sec:konzept}
In den nachfolgenden Kapiteln wird MorphNet mit einer Kombination aus PruneTrain und Net2Net verglichen. MorphNet ist eine Technik, bei der die Struktur des Netzes durch Wiederholtes Verbreitern des Netzes und Verkleinern des Netzes mittels eines Regularisierers gelernt wird.
PruneTrain beschneidet das Netz so, dass unwichtige Gewichte auf Null gesetzt werden mit dem Ziel ganze Kanäle auf Null zu setzen um diese zu Entfernen. Mit der Entfernung von Kanälen und falls alle Kanäle einer Schicht auf Null gesetzt wurde auch ganzen Schichten, soll Trainingszeit gespart werden bei möglichst geringem Accuracy Verlust.



Um diese beiden Methoden vergleichen zu können werden hier zunächst grundlegende Rahmenbedingungen festgelegt.
Beide Methoden bekommen drei verschieden ausgeprägte Netzwerke mit jeweils drei Phasen:
\begin{itemize}
 \item Netz mit $N=[3,3,3]$ und $K=[8,16,32] $
 \item Netz mit $N=[4,4,4]$ und $K=[4,8,16] $
 \item Netz mit $N=[4,4,4]$ und $K=[8,16,32] $
\end{itemize}
Dabei können mehrere Durchgänge der Netzvergrösserung durchlaufen werden. Beschränkt sind diese Durchläufe nicht direkt in der Anzahl, da hier beide Methoden verschieden lange Phasen zwischen den Netzvergrösserungsschritten haben. Um trotzdem zeitlich ein Begrenzung zu haben ist es beiden Methonden nicht erlaubt nach 4 Stunden Laufzeit das Netz nochmal zu vergrössern.

Um für die Netze eine Beschränkung zu haben sind maximal die Anzahl Parameter/Flops erlaubt, die das breite Baseline Netz hat.

MorphNet wird in Kapitel \ref{sec:morphexperimente} evaluiert. In Kapitel \ref{sec:ptexperimente} wird PruneTrain so evaluiert, wie es in der vorgefertigten Implementierung \footnote{\url{https://bitbucket.org/lph\_tools/prunetrain/src/master/}} geschrieben wurde. In Kapitel \ref{sec:ptnew} wird die Erhöhung der Batchgröße bei Beschneiden des Netzes evaluiert.

In Kapitel \ref{sec:net2netexperimente} wird Net2Net evaluiert.
In Kapitel \ref{sec:ptpnet2net} wird die Vorgehensweise zum Kombinieren von PruneTrain und Net2Net beschrieben sowie die Kombination evaluiert.

Der Vergleich der beiden Methoden wird in Kapitel \ref{sec:vergleich} vollzogen.
\todo[inline]{nur wenn noch Zeit übrig: In Kapitel 8 werden die Ergebnisse aus 6.3 auf einem anderen Datensatz verifiziert. auch wenn noch Zeit: Prüfe wie sich die additiven Verfahren auf 6.3 auswirken Zuletzt werden noch additive Verfahren vorgetellt, welche die Trainingszeit zusätzlich minimieren können. Eine dieser Verfahren, welches in Kapitel evaluiert wird, spart Zeit durch die Verwendung von gemischt präzisen Zahlenformaten.
Ein weiteres additives Verfahren in Kapitel  überprüft in wiefern mit Hilfe einer adaptiven Anpassung der Lernrate die Batchgröße sinvoll so angepasst werden kann, dass die ganze GPU genutzt werden kann.}








\cleardoublepage

\chapter{Überblick über die Arbeit}\label{sec:experimente}
Im zweiten Teil dieser Arbeit werden die in Kapitel \ref{sec:wissenschaft} theoretisch betrachteten Methoden praktisch auf einer GPU ausgeführt und evaluiert. Der Überblick über das experimentelle Setup wird in Kapitel \ref{sec:setup} vorgestellt. In Kapitel \ref{sec:konzept} wird das Konzept des praktischen Teils der Arbeit erläutert.

\section{Experimentelles Setup}\label{sec:setup}
\subsection{Hardware}
Die Hardware umfasst einen Server mit 4 GPUs.
Von diesen 4 GPUs haben 2 GPUs jeweils den gleichen Typ:
\begin{itemize}
 \item Geforce GTX 1080 Ti
 \item Geforce RTX 2080 Ti 
\end{itemize}

Beide GPU-Typen arbeiten mit der CUDA Version 10.1. 

Während der Vorbereitung auf diese Experimente hat sich gezeigt, dass Experimente mit einer Geforce GTX 1080 Ti mit den Experimenten der Geforce RTX 2080 Ti nicht vergleichbar sind. Weiterhin lässt sich durch das Verwenden von gemischt präzisen Zahlen nur auf der Geforce RTX 2080 ein Geschwindigkeitsvorteil beim Training feststellen. Aus diesen zwei Gründen wurden alle Experimente auf der Geforce RTX 2080 Ti ausgeführt.  


\subsection{Wahl des Frameworks}

Es wird mit pytorch gearbeitet, da pytorch gegenüber anderen Frameworks eine grössere Flexibiltät erlaubt. Ausserdem ist eine fast vollständige Implementierung von PruneTrain in Pytorch geschrieben. Diese wird im nächsten Kapitel untersucht und soweit erweitert, dass es dem Stand im PruneTrain Paper entspricht.

Pytorch bietet mit cudnn und cuda im Hintergrund gute Möglichkeiten die Trainingszeiten einzelner Epochen zu messen und sie so mit einander zu vergleichen.


\subsection{verwendete Netzarchitektur}\label{sec:archi}
Die PruneTrain Implementierung hat initial mehrere verschiedene Netzarchitekturen zur Auswahl:
\begin{itemize}
 \item AlexNet
 \item ResNet 32/50
 \item vgg 8/11/13/16
 \item mobilenet
\end{itemize}

Diese Auswahl an Netzarchitekturen ist zu umfangreich, um alle diese Architekturen auf den vorgestellten Methoden zu evaluieren. Daher wird im Rahmen dieser Arbeit nur auf ResNet gearbeitet. Diese Entscheidung liegt daran, dass Resnets durch ihre Kurzschlussverbindungen gut mit sehr tiefen Netzstrukturen unmgehen können, ohne grosses Klassifikationsleistungsverluste dank Overfitting. Dies ist vorallem wichtig, wenn das Netz mit Hilfe des Operator für tieferes Netz noch tiefer gemacht werden soll. Die ResNet Struktur wird in der Implementierung so verändert, dass angeben werden kann wie tief das Netz sein soll. Das ResNet wird hier nicht mehr nur mit einer Zahl identifiziert sondern es wird angegeben, wieviele
\begin{itemize}
 \item $s$: Anzahl an Phasen, die das ResNet hat
 \item $N=[n_1, \ldots, n_S]:$ Anzahl von Blöcken pro Phase 
 \item $l$: Anzahl von (Conv+Batch)-Layer pro Block
 \item $[k_1, \ldots,k_S ]:$ Breite der Schichten je Phase
 \item $b$: Boolean Parameter, der angibt ob die Blöcke im Netz die Bootleneck-Eigenschaft haben
\end{itemize}
das jeweilige ResNet hat. Diese Vorgehensweise hat den Vortei, dass für ein im Verlauf tieferes beziehungweise breitere Netz eine Vergleichsmöglichkeit besteht. Dies bedeutet, dass das Netz welches im Verlauf entsteht auch direkt erstellt werden kann.


\subsection{Baseline Netz}\label{sec:baseline}

Um die Ergebnisse der Experimente in den folgenden Kapiteln einschätzen zu können wird ein ResNet, ohne Anpassungen um Trainingszeit zu sparen, trainiert. In Tabelle \ref{tab:baseline} ist die Struktur dieses Netze zu sehen. Das breite Baseline-Netz wird dabei für die Evaluierung des Beschneiden des Netzes verwendet. Das schmallere Baseline-Netz wird für die Evaluierung der Methoden, die das Netz breiter machen verwendet.

Das Netz hat drei Phasen $(s=3)$, wobei jeder der Phasen 5 Blöcke hat $(N=[5,5,5])$. Pro Basisblock sind zwei (Conv+Batch)-Schichten vorhanden $(l=2)$. Bei einem Übergangsblock, der als erster Block in einer neuen Phase bei einer Vergrösserung der Bereit beim Phasenübergang genutzt wird ist eine (Conv+Batch)-Schicht mehr vorhanden. Eine grafische Darstellung der Blöcke ist in Abbildung \ref{abb:blocks} zu sehen.




\begin{figure}[]
   \begin{minipage}[b]{.4\linewidth} % [b] => Ausrichtung an \caption
      \includegraphics[width=0.8\linewidth]{KapitelPartB/Images/Basisblock.png}
      \caption{Basisblock}
   \end{minipage}
   \hspace{.1\linewidth}% Abstand zwischen Bilder
   \begin{minipage}[b]{.4\linewidth} % [b] => Ausrichtung an \caption
      \includegraphics[width=0.8\linewidth]{KapitelPartB/Images/Ubergangsblock.png}
      \caption{Übergangsblock}
   \end{minipage}
   \caption{Grafische Darstellung Basis- und Übergangsblock}
   \label{abb:blocks}
\end{figure}

\begin{table}[]
\begin{tabular}{|l|l|l|l|l|l|}
\hline
      &                & \multicolumn{2}{c|}{breites Baseline-Netz} &\multicolumn{2}{c|}{schmalles Baseline-Netz} \\ 
Phase & Schicht/Block  & \#Eingangs- & \#Ausgangs-       & \#Eingangs- & \#Ausgangs-    \\
      &                & \multicolumn{2}{c|}{kanäle}     & \multicolumn{2}{c|}{kanäle}  \\ \hline
      & Conv 1 + Bn 1  & 3                & 16           & 3           & 8              \\ \hline \hline
1     & Basisblock     & 16               & 16           & 8           & 8              \\ \hline
      & Basisblock     & 16               & 16           & 8           & 8              \\ \hline
      & Basisblock     & 16               & 16           & 8           & 8              \\ \hline
      & Basisblock     & 16               & 16           & 8           & 8              \\ \hline
      & Basisblock     & 16               & 16           & 8           & 8              \\ \hline \hline
2     & Übergangsblock & 16               & 32           & 8           & 16             \\ \hline
      & Basisblock     & 32               & 32           & 16          & 16             \\ \hline
      & Basisblock     & 32               & 32           & 16          & 16             \\ \hline
      & BasisBlock     & 32               & 32           & 16          & 16             \\ \hline
      & BasisBlock     & 32               & 32           & 16          & 16             \\ \hline \hline
3     & Übergangsblock & 32               & 64           & 16          & 32             \\ \hline
      & Basisblock     & 64               & 64           & 32          & 32             \\ \hline
      & Basisblock     & 64               & 64           & 32          & 32             \\ \hline
      & Basisblock     & 64               & 64           & 32          & 32             \\ \hline
      & Basisblock     & 64               & 64           & 32          & 32             \\ \hline \hline
      & Linear         & 64               & 10           & 32          & 10             \\ \hline
\end{tabular}
\caption{Struktur des Netzes}
\label{tab:baseline}
\end{table}


\subsubsection{Evaluierung des breiteren Baseline-Netzes}
\color{vermi}
Das Training wird über 180 Epochen durchgeführt. Es werden 5 Experimente durchgeführt. Dabei ergibt sich der in Abbildung \ref{abb:baseAcc1} gezeigte Verlauf der Validierungs-Accuracy für Experiment vier. Bei diesem Training wurde über die gesamten 180 Epochen mit einer Lernrate von 0.1 trainiert. Mit einem Ergebnis von durchschnittlich 81.11 \% über fünf Experimente ist diese Ergebnis leider nicht zufriedenstellend. 

Eine Verkleinerung der Lernrate kann dieses Ergebnis verbessern \cite{CNNBook}. In Abbildung \ref{abb:baseAcc2} wird dargestellt, wie sich der Verlauf ändert durch eine Anpassung der Lernrate bei Epoche 93 und 150. In diesen beiden Epochen wird die Lernrate jeweils auf ein Zehntel verkleinert. 

In Abbildung \ref{abb:baseAcc3} ist ein Boxplot dargestellt, der die Accuracy von jeweils fünf Experimenten mit oder ohne Anpassung der Lernrate vergleicht. Es ergibt sich eine deutliche Verbesserung der Accuracy des Baseline-Netzes von 81.24 \% auf 91.89 \% durch diese Anpassung. Aufgrund dieser Verbesserung werden die Experimente mit Anpassung der Lernrate als Grundlage zum Vergleich mit den Experimenten in den weiteren Kapiteln herangezogen. Sie werden in diesen Kapitel nur mit Baseline-Netz bezeichnet.

\begin{figure}
     \centering
     \subfloat[][]{\includegraphics[width=.49\textwidth]{KapitelPartB/Images/BaseAcc1.png}\label{abb:baseAcc1}}
     \subfloat[][]{\includegraphics[width=.49\textwidth]{KapitelPartB/Images/BaseAcc.png}\label{abb:baseAcc2}}\\
     \subfloat[][]{\includegraphics[width=.49\textwidth]{KapitelPartB/Images/BaseAcc3.png}\label{abb:baseAcc3}}
     \caption{Vergleich zwischen (a) Baseline-Netz ohne Anpassung der Lernrate und (b) Baseline-Netz mit Anpassung der Lernrate in Epoche 93 und 150. (c)  Boxplot der Accuracys}
     \label{abb:BaseAcc}
\end{figure}

Ein weiteres Vergleichskriterium neben der Accuracy sind die durchschnittlichen Trainingszeiten pro Epoche. Für jedes Experiment wird die durchschnittliche Dauer einer Trainingsepoche mittel des arithmetiscen Mittels berechnet.
Die Durchschnittswerte über alle 180 Epoche sind für die zehn Baseline Experimente sind in Tabelle \ref{tab:baselineTime} aufgelistet. Die Durchschnittswerte liegen sehr nah beeinander, der Unterschied zwischen dem grössten und dem kleinsten Durchschnittswert liegt bei 0,28. Damit ergeben sich zwischen den zehn Experimenten keine signifikante Unterschiede. Es wird daher mit Experiment vier eines der Experimente mit Anpassung der Lernrate ausgesucht um für die folgenden Experimente/ Kapitel als Vergleich zu dienen.
\begin{table}[h]
\begin{tabular}{|l|l|l|l|l|l|l|l|l|l|l|} \hline
           & \multicolumn{5}{c|}{Experimente ohne Anpassung}&\multicolumn{5}{c|}{Experimente mit Anpassung} \\
           &\multicolumn{5}{c|}{der Lernrate} &\multicolumn{5}{c|}{der Lernrate}\\
           & 1       & 2      & 3      & 4       & 5       & 1      & 2     & 3      & 4     & 5  \\ \hline 
$\mu$      & 19,49   & 19,53  & 19,50  & 19,48   & 19,84   & 19,57  & 19,56 & 19,53  & 19.53 & 19.62 \\ \hline
\end{tabular}
\caption{Tabelle für Durchschnittswerte und Standardabweichungen der Trainingszeiten der Experimente}
\label{tab:baselineTime}
\end{table}
\color{black}
\subsubsection{Evaluierung des schmalleren Baseline-Netzes}
\color{blue1}
Das schmallere Baseline-Netz wird verwendet, um für MorphNet und Net2Net eine schmalle Variante zu haben. Der Grund hierfür ist, dass bei einer durchschnittlichen Accuracy von 93,19 \% des breiten Baseline-Netzes nicht mehr viel Raum für Verbesserungen bleibt. In Abbildung \ref{abb:baseAcc3} sind die Experimente für das breite und schmalle Baseline-Netz mit der Anpassung der Lernrate gegenüber gestellt. Der Unterschied vom breiten zum schmallen Netz ist ein Accuracy-Verlust von 3,1 \%.    


In Abbildung \ref{abb:baseAccS1} ist abgebildet, wie sich die Accuracy für das schmallere Netz verhält, bei der gleichen Anpassung der Lernrate wie beim breiteren Baseline-Netzen. Der Unterschied in der Accuracy zwischen dem schmallen und breiten Baseline-Netz ist in Abbildung \ref{abb:baseAccS2} abgebildet.


\begin{figure}
     \centering
     \subfloat[][]{ \includegraphics[width=0.5\textwidth]{KapitelPartB/Images/baseAccS.png}\label{abb:baseAccS1}}
     \subfloat[][]{\includegraphics[width=.49\textwidth]{KapitelPartB/Images/baseAccSHisto.png}\label{abb:baseAccS2}}
     \caption{}
     \label{abb:BaseAccS}
\end{figure}


\color{black}

\section{Konzept}\label{sec:konzept}
In den nachfolgenden Kapiteln wird MorphNet mit einer Kombination aus PruneTrain und Net2Net verglichen. MorphNet ist eine Technik, bei der die Struktur des Netzes durch Wiederholtes Verbreitern des Netzes und Verkleinern des Netzes mittels eines Regularisierers gelernt wird.
PruneTrain beschneidet das Netz so, dass unwichtige Gewichte auf Null gesetzt werden mit dem Ziel ganze Kanäle auf Null zu setzen um diese zu Entfernen. Mit der Entfernung von Kanälen und falls alle Kanäle einer Schicht auf Null gesetzt wurde auch ganzen Schichten, soll Trainingszeit gespart werden bei möglichst geringem Accuracy Verlust.



Um diese beiden Methoden vergleichen zu können werden hier zunächst grundlegende Rahmenbedingungen festgelegt.
Beide Methoden bekommen drei verschieden ausgeprägte Netzwerke mit jeweils drei Phasen:
\begin{itemize}
 \item Netz mit $N=[3,3,3]$ und $K=[8,16,32] $
 \item Netz mit $N=[4,4,4]$ und $K=[4,8,16] $
 \item Netz mit $N=[4,4,4]$ und $K=[8,16,32] $
\end{itemize}
Dabei können mehrere Durchgänge der Netzvergrösserung durchlaufen werden. Beschränkt sind diese Durchläufe nicht direkt in der Anzahl, da hier beide Methoden verschieden lange Phasen zwischen den Netzvergrösserungsschritten haben. Um trotzdem zeitlich ein Begrenzung zu haben ist es beiden Methonden nicht erlaubt nach 4 Stunden Laufzeit das Netz nochmal zu vergrössern.

Um für die Netze eine Beschränkung zu haben sind maximal die Anzahl Parameter/Flops erlaubt, die das breite Baseline Netz hat.

MorphNet wird in Kapitel \ref{sec:morphexperimente} evaluiert. In Kapitel \ref{sec:ptexperimente} wird PruneTrain so evaluiert, wie es in der vorgefertigten Implementierung \footnote{\url{https://bitbucket.org/lph\_tools/prunetrain/src/master/}} geschrieben wurde. In Kapitel \ref{sec:ptnew} wird die Erhöhung der Batchgröße bei Beschneiden des Netzes evaluiert.

In Kapitel \ref{sec:net2netexperimente} wird Net2Net evaluiert.
In Kapitel \ref{sec:ptpnet2net} wird die Vorgehensweise zum Kombinieren von PruneTrain und Net2Net beschrieben sowie die Kombination evaluiert.

Der Vergleich der beiden Methoden wird in Kapitel \ref{sec:vergleich} vollzogen.
\todo[inline]{nur wenn noch Zeit übrig: In Kapitel 8 werden die Ergebnisse aus 6.3 auf einem anderen Datensatz verifiziert. auch wenn noch Zeit: Prüfe wie sich die additiven Verfahren auf 6.3 auswirken Zuletzt werden noch additive Verfahren vorgetellt, welche die Trainingszeit zusätzlich minimieren können. Eine dieser Verfahren, welches in Kapitel evaluiert wird, spart Zeit durch die Verwendung von gemischt präzisen Zahlenformaten.
Ein weiteres additives Verfahren in Kapitel  überprüft in wiefern mit Hilfe einer adaptiven Anpassung der Lernrate die Batchgröße sinvoll so angepasst werden kann, dass die ganze GPU genutzt werden kann.}








\section{Verringerung der für Berechnungen nötige Zeit}

Die Zeit, die ein Convolutional Layer braucht um berechnet zu werden hängt ab von:
\todo{Fehlt hier noch etwas?}
\begin{itemize}
 \item der Filtergr\"osse
 \item der Bildgr\"osse
 \item dem verwendeten Zahlenformat
\end{itemize}
Beim Verändern der Filter- oder der Bildgr\"osse, um Trainingszeit zu sparen, ver\"andert sich auch die Erkennungsleistung \todo{cite}. Dies ist beim Verändern des verwendeten Zahlenformats nicht umbedingt gegeben. Standardformat ist eine 32 Bit Gleitkommazahl. Die einfachste Methode hier Trainingszeit zu sparen ist das Halbieren der Bitanzahl auf 16 Bit. Eine weitere Methode ist das Benutzen von 16 Bit Dynamischen Festkommazahlen.
Die beiden alternativen Methoden haben unterschiedliche Anforderungen an die Ausführungsplattform. Diese Anforderungen und die Besonderheiten der beiden Verfahren werden in den folgenden zwei Unterkapiteln näher beleuchtet.


\subsection{Berechnung mit 16 Bit Gleitkomma}

Die 16 Bit Gleitkommazahl unterscheidet sich nicht nur in der Länge von der 32 Bit Zahl sondern aus der unterschiedlichen Länge erwachsen Unterschiede in den darstellbaren Zahlen. In Tabelle \todo{ref} sind diese Unterschiede dargestellt.





Diese Nachteile von 16 Bit Gleitkommazahlen können durch drei Techniken abgemeildert oder sogar komplett aufgehoben werden:
\begin{itemize}
 \item 32 Bit Mastergewichte und Updates
 \item Sklaierung der Loss-Funktion
 \item Arithmetische Präzision 
\end{itemize}

Diese drei Techniken werden in den drei folgenden Unterkpaiteln behandelt.

\subsubsection{32 Bit Mastergewichte und Updates}

Beim Trainieren von neuronalen Netzwerken mit 16 Bit Gleitkommazahlen werden die Gewichte, Aktivierungen und Gradienten im 16 Bit Format gespeichert. Die Speicherung der Gewichte als 32 Bit Mastergewichte hat zwei mögliche Erklärungen, die aber nicht immer zutreffen müssen. 

Um nach einem Forward Druchlauf des Netzes die Gewichte abzudaten wird ein Gradientenabstiegsverfahren benutzt. Hierbei werden die Gradienten der Gewichte berechnet. Um für die Funktion, die das CNN approximiert einen besseren Approximationserfolg zu erlangen wird dann dieser Gradient mit der Lernrate multipliziert. Wird dieses Produkt in 16 Bit abgespeichert, so ist in viele Fällen das Produkt der beiden Zahlen gleich Null. Dies liegt an der Taqtsache, dass wie in Tabelle \todo{ref} zu sehen ist die kleinste darstellbare Zahl in 16 Bit wesentlich grösser ist als in 32 Bit.


Der zweite Grund wieso man Mastergewichte brauchen könnte ist die Tatsache, dass bei grossen Gewichten die Länge der Mantisse nicht ausreicht, um sowohl das Gewicht als auch das zu  addierende Update zu speichern.

Aus den beiden Gründen wird das in Abbildung \todo{ref} gezeigte Schema zum Trainieren einer Schicht mit gemischt präzisen Gleitkommazahlen benutzt.

\missingfigure{Schema}

\subsubsection{Sklaierung der Loss-Funktion}

\subsubsection{Arithmetische Präzision}


\subsection{Berechnung mit 16 Bit Dynamischen Festkommazahlen}


Quelle: \cite{FPGpu}


\chapter{Beschleunigung der Berechnung des Gradientenabstiegsverfahren}
Bei der Beschleunigung der Berechnung des Gradientenabstiegsverfahren gibt es vier verschiedene publizierte Herangehensweisen:
\begin{itemize}
 \item Accelerating CNN Training by Sparsifying Activation Gradients
 \item Weight Normalization: A Simple Reparameterization
to Accelerate Training of Deep Neural Networks
 \item Accelerating Deep Neural Network Training with Inconsistent Stochastic Gradient Descent
 \item Accelerated CNN Training Through Gradient Approximation 
\end{itemize}


\section{Accelerating CNN Training by Sparsifying Activation Gradients}

Funktioniert nur auf Toy-Benchmarks


\section{Weight Normalization: A Simple Reparameterization
to Accelerate Training of Deep Neural Networks}



\section{Accelerating Deep Neural Network Training with Inconsistent Stochastic Gradient Descent}


\section{Accelerated CNN Training Through Gradient Approximation }


\chapter{Verfahren um weniger Trainingsdaten zu verwenden}

\section{Stochastisches Pooling}




\section{Lernen von Struktur und Stärke von CNNs}


\chapter{Strukturelle Veränderungen zur Beschleunigung des Trainings}

\section{Pruning um Trainingszeit zu minimieren}
Pruning ist eine Technik, die entwickelt wurde, um die Inferenzzeit eines neuronalen Netzwerks zu reduzieren. Das Pruningverfahren wird auf das bereits trainierte Netz angewendet. Dabei wird entschieden, welche Gewichte nur einen minimalen Effekt auf das Klassifikationsergebnis haben um diese zu entfernen.

Aktueller Gegenstand der Forschung ist hier die Frage, ob diese kleineren Netzwerke nicht bereits ab Epoche Null trainiert werden können, um so Trainingszeit zu sparen. Dieser Ansatz wurde in verschiedenen Veröffentlichungen untersucht:
\begin{itemize}
 \item Prune Train 
 \item The Lottery Ticket Hypothesis
\end{itemize}
Zunächst werden die einzelen Verfahren erläutert, um sie danach miteinander zu vergleichen.

\subsection{Prune Train}
Prune Train fügt einen Normalisierungsterm zur Loss-Funktion des Netzwerkes hinzu. Dies geschieht, damit der Optimierungsprozess dazu gezwungen wird möglichst kleine Gewichte zu wählen. Durch diesen Prozess wird aus dem dense Netz ein sparse Netz. Dieses sparse Netz sorgt allerdings noch nicht für weniger Zeitbedarf einer Trainingsepoche, da für ein Sparse aufwändige Datenindextechniken notwendig sind. Daher wird bei diesem Verfahren das Netz rekonfiguiert um das Modell kleiner und die Struktur wieder dense zu machen.
Dabei hat Prune Train drei zentrale Optimierungsverfahren:
\begin{itemize}
   \item eine systematische Methode zur Berechnung des group lasso Regularisierung Sanktions Koeffizienten beim Beginn des Trainings.
   \item Kanal union, ein Speicheraufruf kosteneffizientes und Index-freies Kanal Pruning Verfahren für moderne CNNs mit Kurzschlussverbindungen.
   \item Ein dynamische Mini-Batch Adjustment, dass die Größe des Mini-Batch anpasst. Dies geschieht durch beobachten des Speicherkapazitätgebrauchs einer Trainingsiteration nach jeder Pruning reconfiguration.
\end{itemize} 

Der group lasso Regularisierung Sanktions Koeffizienten ist ein Hyperparameter, der einen Trade-off  zwischen der Modellgröße und der Accuracy bildet.
Voherige Arbeiten suchen nach einem geeignetem Sanktionsmaß, was das Einbeziehen des Prunings vom Anfang des Trainings sehr teuer macht.
Unser Mechanismus kontrolliert die Group lasso Regularisierungstärke und erreicht eine hohe Modellpruningrate mit nur einem kleinen Einfluss auf die Accuracy bei nur einem Trainingsdurchlauf.
Kurzschlussverbindungen werden in modernen CNNs häufig genutzt.
Prunning aller genullten Kanäle solcher CNNs brauchen regelmässige Tensor Umordnung um die Kanalindizes zwischen den Schichten zu matchen. Dies vermindert die Performance.
Diese Umordnung wird durch den Channel Union Algorithmus vermieden. Daher folgt eine 1.9 fache Beschleunigung des Convolutional Layetrs.

Dynmaisches Mini Batch Adjustment kompensiert die verminderte Datenparallelität aufgrund des kleineren geprunten Modells durch Erhöhung der Mini-Batch Größe.
Dies sorgt sowohl für bessere Ausnutzung der Hardware ressourcen als auch zur Reduzierung des KOmmunikation overheads durch eine Verminderte Modell Update Frequenz. Beim Erhöhen der Mini-Batch Größe wird auch die Lernrate mit demselben Verhältnis erhöht, um die Accuracy nicht zu verändern.
 $$ \underset{min}{W} \left( \frac{1}{N} \sum_{i=1}^{N} l(y_i,f(x_i, W)) + \sum_{g=1}^{G} \lambda_g \cdot || W_g ||_2 \right) $$
 \begin{itemize}
  \item $ \frac{1}{N} \sum_{i=1}^{N} l(y_i,f(x_i, W))$ Standard-Kreuzentropie
  \item $\sum_{g=1}^{G} \lambda_g \cdot || W_g ||_2 $ group lasso Regulierungsterm
  \item $f(x_i)$ Vorhersage des Netzwerks auf Eingabe $x_i$
  \item $W$ Gewichte
  \item $l$ Verlustfunktion der Klassifikation und Grundwahrheit $y_i$
  \item $N$ Minibatchgröße
  \item $G$ Zahl von Gruppen
  \item $\lambda$ Verdünnungskoeffizient
\end{itemize}

$$ \lambda \cdot \sum_{l=1}^{L} \left( \sum_{c_l=1}^{C_l} || W_{c_l,:,:,:} ||_2 + \sum_{k_l=1}^{K_l} || W_{:,k_l,:,:}||_2 \right) $$ 
Design eines speziellen Groupo Lasso Regulierers, der Gewichte jedes Kanals (Input oder Output) und jeder Schicht gruppiert. $\lambda$ wird als einziger globaler Regularisierungsfaktor gewählt, da so der Fokus auf dem Vermindern der Rechenzeit liegt und nicht auf der Modellgröße. Dies hat zur Folge, dass vorallem große Features verdünnt werden, was zu einer größeren Verminderung der Rechenleistung führt.

Um die Lasso Group Regularisierung vom Anfang des Trainings zu benutzen sollteder Koeffizient $\lambda$ sinnvoll gewählt werden. Dies sorgt für eine hohe Vorhersageaccuracy und einer hohen Pruning Rate. Um zeitintensives Hyperparametertuning zu vermeiden wird hier eine neue Methode eingeführt:
 
 $$LPR=\frac{\lambda \sum_{g}^{G}|| W_{g,:} ||}{l(y_i,f(x_i,W))+\lambda \sum_{g}^{G}||W_{g,:} ||)} $$
 
 Berechnet wird dies durch setzen von zufälligen Werten, mit denen die Gewichte initialisiert werden. LPR wird einmal berechnet und dann bis zum Ende weiter benutzt.

 
 Nach jedem solchen Intervall werden Input und Outputkanäle die 0 sind gepruned. Um ein Missverhältnis zwischen den Dimensionen zu verhindern wird nur die Verbindung von 2 verdünnten Kanälen von 2 aufeinanderfolgenden Schichten gepruned. Alle Trainingsvariablen bleiben gleich.
 Das Reconfigurationsintervall ist ist der einzige zusätzliche Hyperparameter. Zu gross gewählt würde der Intervallparameter zu wenig Zeitverbesserung bringen. Zu klein gewählt könnte er die Lernqualität beeinflussen.
4 Matriken zur Evaluierung: Training und Inference FLOPs, gemessenen Trainingszeit, und Validierungsaccuracy. 




\section{Net 2 Net}


\section{Kernel rescaling}


\section{Resource Aware Layer Replacement}


\section{Additive Methoden}
Die Methoden in diesem Kapitel beeinflussen die Trainingszeit nicht direkt, sondern helfen die Folgen anderer Verfahren abzumildern.

\subsection{Ghost Batch Normalization}





\cleardoublepage
\chapter{Überblick über die Arbeit}\label{sec:experimente}
Im zweiten Teil dieser Arbeit werden die in Kapitel \ref{sec:wissenschaft} theoretisch betrachteten Methoden praktisch auf einer GPU ausgeführt und evaluiert. Der Überblick über das experimentelle Setup wird in Kapitel \ref{sec:setup} vorgestellt. In Kapitel \ref{sec:konzept} wird das Konzept des praktischen Teils der Arbeit erläutert.

\section{Experimentelles Setup}\label{sec:setup}
\subsection{Hardware}
Die Hardware umfasst einen Server mit 4 GPUs.
Von diesen 4 GPUs haben 2 GPUs jeweils den gleichen Typ:
\begin{itemize}
 \item Geforce GTX 1080 Ti
 \item Geforce RTX 2080 Ti 
\end{itemize}

Beide GPU-Typen arbeiten mit der CUDA Version 10.1. 

Während der Vorbereitung auf diese Experimente hat sich gezeigt, dass Experimente mit einer Geforce GTX 1080 Ti mit den Experimenten der Geforce RTX 2080 Ti nicht vergleichbar sind. Weiterhin lässt sich durch das Verwenden von gemischt präzisen Zahlen nur auf der Geforce RTX 2080 ein Geschwindigkeitsvorteil beim Training feststellen. Aus diesen zwei Gründen wurden alle Experimente auf der Geforce RTX 2080 Ti ausgeführt.  


\subsection{Wahl des Frameworks}

Es wird mit pytorch gearbeitet, da pytorch gegenüber anderen Frameworks eine grössere Flexibiltät erlaubt. Ausserdem ist eine fast vollständige Implementierung von PruneTrain in Pytorch geschrieben. Diese wird im nächsten Kapitel untersucht und soweit erweitert, dass es dem Stand im PruneTrain Paper entspricht.

Pytorch bietet mit cudnn und cuda im Hintergrund gute Möglichkeiten die Trainingszeiten einzelner Epochen zu messen und sie so mit einander zu vergleichen.


\subsection{verwendete Netzarchitektur}\label{sec:archi}
Die PruneTrain Implementierung hat initial mehrere verschiedene Netzarchitekturen zur Auswahl:
\begin{itemize}
 \item AlexNet
 \item ResNet 32/50
 \item vgg 8/11/13/16
 \item mobilenet
\end{itemize}

Diese Auswahl an Netzarchitekturen ist zu umfangreich, um alle diese Architekturen auf den vorgestellten Methoden zu evaluieren. Daher wird im Rahmen dieser Arbeit nur auf ResNet gearbeitet. Diese Entscheidung liegt daran, dass Resnets durch ihre Kurzschlussverbindungen gut mit sehr tiefen Netzstrukturen unmgehen können, ohne grosses Klassifikationsleistungsverluste dank Overfitting. Dies ist vorallem wichtig, wenn das Netz mit Hilfe des Operator für tieferes Netz noch tiefer gemacht werden soll. Die ResNet Struktur wird in der Implementierung so verändert, dass angeben werden kann wie tief das Netz sein soll. Das ResNet wird hier nicht mehr nur mit einer Zahl identifiziert sondern es wird angegeben, wieviele
\begin{itemize}
 \item $s$: Anzahl an Phasen, die das ResNet hat
 \item $N=[n_1, \ldots, n_S]:$ Anzahl von Blöcken pro Phase 
 \item $l$: Anzahl von (Conv+Batch)-Layer pro Block
 \item $[k_1, \ldots,k_S ]:$ Breite der Schichten je Phase
 \item $b$: Boolean Parameter, der angibt ob die Blöcke im Netz die Bootleneck-Eigenschaft haben
\end{itemize}
das jeweilige ResNet hat. Diese Vorgehensweise hat den Vortei, dass für ein im Verlauf tieferes beziehungweise breitere Netz eine Vergleichsmöglichkeit besteht. Dies bedeutet, dass das Netz welches im Verlauf entsteht auch direkt erstellt werden kann.


\subsection{Baseline Netz}\label{sec:baseline}

Um die Ergebnisse der Experimente in den folgenden Kapiteln einschätzen zu können wird ein ResNet, ohne Anpassungen um Trainingszeit zu sparen, trainiert. In Tabelle \ref{tab:baseline} ist die Struktur dieses Netze zu sehen. Das breite Baseline-Netz wird dabei für die Evaluierung des Beschneiden des Netzes verwendet. Das schmallere Baseline-Netz wird für die Evaluierung der Methoden, die das Netz breiter machen verwendet.

Das Netz hat drei Phasen $(s=3)$, wobei jeder der Phasen 5 Blöcke hat $(N=[5,5,5])$. Pro Basisblock sind zwei (Conv+Batch)-Schichten vorhanden $(l=2)$. Bei einem Übergangsblock, der als erster Block in einer neuen Phase bei einer Vergrösserung der Bereit beim Phasenübergang genutzt wird ist eine (Conv+Batch)-Schicht mehr vorhanden. Eine grafische Darstellung der Blöcke ist in Abbildung \ref{abb:blocks} zu sehen.




\begin{figure}[]
   \begin{minipage}[b]{.4\linewidth} % [b] => Ausrichtung an \caption
      \includegraphics[width=0.8\linewidth]{KapitelPartB/Images/Basisblock.png}
      \caption{Basisblock}
   \end{minipage}
   \hspace{.1\linewidth}% Abstand zwischen Bilder
   \begin{minipage}[b]{.4\linewidth} % [b] => Ausrichtung an \caption
      \includegraphics[width=0.8\linewidth]{KapitelPartB/Images/Ubergangsblock.png}
      \caption{Übergangsblock}
   \end{minipage}
   \caption{Grafische Darstellung Basis- und Übergangsblock}
   \label{abb:blocks}
\end{figure}

\begin{table}[]
\begin{tabular}{|l|l|l|l|l|l|}
\hline
      &                & \multicolumn{2}{c|}{breites Baseline-Netz} &\multicolumn{2}{c|}{schmalles Baseline-Netz} \\ 
Phase & Schicht/Block  & \#Eingangs- & \#Ausgangs-       & \#Eingangs- & \#Ausgangs-    \\
      &                & \multicolumn{2}{c|}{kanäle}     & \multicolumn{2}{c|}{kanäle}  \\ \hline
      & Conv 1 + Bn 1  & 3                & 16           & 3           & 8              \\ \hline \hline
1     & Basisblock     & 16               & 16           & 8           & 8              \\ \hline
      & Basisblock     & 16               & 16           & 8           & 8              \\ \hline
      & Basisblock     & 16               & 16           & 8           & 8              \\ \hline
      & Basisblock     & 16               & 16           & 8           & 8              \\ \hline
      & Basisblock     & 16               & 16           & 8           & 8              \\ \hline \hline
2     & Übergangsblock & 16               & 32           & 8           & 16             \\ \hline
      & Basisblock     & 32               & 32           & 16          & 16             \\ \hline
      & Basisblock     & 32               & 32           & 16          & 16             \\ \hline
      & BasisBlock     & 32               & 32           & 16          & 16             \\ \hline
      & BasisBlock     & 32               & 32           & 16          & 16             \\ \hline \hline
3     & Übergangsblock & 32               & 64           & 16          & 32             \\ \hline
      & Basisblock     & 64               & 64           & 32          & 32             \\ \hline
      & Basisblock     & 64               & 64           & 32          & 32             \\ \hline
      & Basisblock     & 64               & 64           & 32          & 32             \\ \hline
      & Basisblock     & 64               & 64           & 32          & 32             \\ \hline \hline
      & Linear         & 64               & 10           & 32          & 10             \\ \hline
\end{tabular}
\caption{Struktur des Netzes}
\label{tab:baseline}
\end{table}


\subsubsection{Evaluierung des breiteren Baseline-Netzes}
\color{vermi}
Das Training wird über 180 Epochen durchgeführt. Es werden 5 Experimente durchgeführt. Dabei ergibt sich der in Abbildung \ref{abb:baseAcc1} gezeigte Verlauf der Validierungs-Accuracy für Experiment vier. Bei diesem Training wurde über die gesamten 180 Epochen mit einer Lernrate von 0.1 trainiert. Mit einem Ergebnis von durchschnittlich 81.11 \% über fünf Experimente ist diese Ergebnis leider nicht zufriedenstellend. 

Eine Verkleinerung der Lernrate kann dieses Ergebnis verbessern \cite{CNNBook}. In Abbildung \ref{abb:baseAcc2} wird dargestellt, wie sich der Verlauf ändert durch eine Anpassung der Lernrate bei Epoche 93 und 150. In diesen beiden Epochen wird die Lernrate jeweils auf ein Zehntel verkleinert. 

In Abbildung \ref{abb:baseAcc3} ist ein Boxplot dargestellt, der die Accuracy von jeweils fünf Experimenten mit oder ohne Anpassung der Lernrate vergleicht. Es ergibt sich eine deutliche Verbesserung der Accuracy des Baseline-Netzes von 81.24 \% auf 91.89 \% durch diese Anpassung. Aufgrund dieser Verbesserung werden die Experimente mit Anpassung der Lernrate als Grundlage zum Vergleich mit den Experimenten in den weiteren Kapiteln herangezogen. Sie werden in diesen Kapitel nur mit Baseline-Netz bezeichnet.

\begin{figure}
     \centering
     \subfloat[][]{\includegraphics[width=.49\textwidth]{KapitelPartB/Images/BaseAcc1.png}\label{abb:baseAcc1}}
     \subfloat[][]{\includegraphics[width=.49\textwidth]{KapitelPartB/Images/BaseAcc.png}\label{abb:baseAcc2}}\\
     \subfloat[][]{\includegraphics[width=.49\textwidth]{KapitelPartB/Images/BaseAcc3.png}\label{abb:baseAcc3}}
     \caption{Vergleich zwischen (a) Baseline-Netz ohne Anpassung der Lernrate und (b) Baseline-Netz mit Anpassung der Lernrate in Epoche 93 und 150. (c)  Boxplot der Accuracys}
     \label{abb:BaseAcc}
\end{figure}

Ein weiteres Vergleichskriterium neben der Accuracy sind die durchschnittlichen Trainingszeiten pro Epoche. Für jedes Experiment wird die durchschnittliche Dauer einer Trainingsepoche mittel des arithmetiscen Mittels berechnet.
Die Durchschnittswerte über alle 180 Epoche sind für die zehn Baseline Experimente sind in Tabelle \ref{tab:baselineTime} aufgelistet. Die Durchschnittswerte liegen sehr nah beeinander, der Unterschied zwischen dem grössten und dem kleinsten Durchschnittswert liegt bei 0,28. Damit ergeben sich zwischen den zehn Experimenten keine signifikante Unterschiede. Es wird daher mit Experiment vier eines der Experimente mit Anpassung der Lernrate ausgesucht um für die folgenden Experimente/ Kapitel als Vergleich zu dienen.
\begin{table}[h]
\begin{tabular}{|l|l|l|l|l|l|l|l|l|l|l|} \hline
           & \multicolumn{5}{c|}{Experimente ohne Anpassung}&\multicolumn{5}{c|}{Experimente mit Anpassung} \\
           &\multicolumn{5}{c|}{der Lernrate} &\multicolumn{5}{c|}{der Lernrate}\\
           & 1       & 2      & 3      & 4       & 5       & 1      & 2     & 3      & 4     & 5  \\ \hline 
$\mu$      & 19,49   & 19,53  & 19,50  & 19,48   & 19,84   & 19,57  & 19,56 & 19,53  & 19.53 & 19.62 \\ \hline
\end{tabular}
\caption{Tabelle für Durchschnittswerte und Standardabweichungen der Trainingszeiten der Experimente}
\label{tab:baselineTime}
\end{table}
\color{black}
\subsubsection{Evaluierung des schmalleren Baseline-Netzes}
\color{blue1}
Das schmallere Baseline-Netz wird verwendet, um für MorphNet und Net2Net eine schmalle Variante zu haben. Der Grund hierfür ist, dass bei einer durchschnittlichen Accuracy von 93,19 \% des breiten Baseline-Netzes nicht mehr viel Raum für Verbesserungen bleibt. In Abbildung \ref{abb:baseAcc3} sind die Experimente für das breite und schmalle Baseline-Netz mit der Anpassung der Lernrate gegenüber gestellt. Der Unterschied vom breiten zum schmallen Netz ist ein Accuracy-Verlust von 3,1 \%.    


In Abbildung \ref{abb:baseAccS1} ist abgebildet, wie sich die Accuracy für das schmallere Netz verhält, bei der gleichen Anpassung der Lernrate wie beim breiteren Baseline-Netzen. Der Unterschied in der Accuracy zwischen dem schmallen und breiten Baseline-Netz ist in Abbildung \ref{abb:baseAccS2} abgebildet.


\begin{figure}
     \centering
     \subfloat[][]{ \includegraphics[width=0.5\textwidth]{KapitelPartB/Images/baseAccS.png}\label{abb:baseAccS1}}
     \subfloat[][]{\includegraphics[width=.49\textwidth]{KapitelPartB/Images/baseAccSHisto.png}\label{abb:baseAccS2}}
     \caption{}
     \label{abb:BaseAccS}
\end{figure}


\color{black}

\section{Konzept}\label{sec:konzept}
In den nachfolgenden Kapiteln wird MorphNet mit einer Kombination aus PruneTrain und Net2Net verglichen. MorphNet ist eine Technik, bei der die Struktur des Netzes durch Wiederholtes Verbreitern des Netzes und Verkleinern des Netzes mittels eines Regularisierers gelernt wird.
PruneTrain beschneidet das Netz so, dass unwichtige Gewichte auf Null gesetzt werden mit dem Ziel ganze Kanäle auf Null zu setzen um diese zu Entfernen. Mit der Entfernung von Kanälen und falls alle Kanäle einer Schicht auf Null gesetzt wurde auch ganzen Schichten, soll Trainingszeit gespart werden bei möglichst geringem Accuracy Verlust.



Um diese beiden Methoden vergleichen zu können werden hier zunächst grundlegende Rahmenbedingungen festgelegt.
Beide Methoden bekommen drei verschieden ausgeprägte Netzwerke mit jeweils drei Phasen:
\begin{itemize}
 \item Netz mit $N=[3,3,3]$ und $K=[8,16,32] $
 \item Netz mit $N=[4,4,4]$ und $K=[4,8,16] $
 \item Netz mit $N=[4,4,4]$ und $K=[8,16,32] $
\end{itemize}
Dabei können mehrere Durchgänge der Netzvergrösserung durchlaufen werden. Beschränkt sind diese Durchläufe nicht direkt in der Anzahl, da hier beide Methoden verschieden lange Phasen zwischen den Netzvergrösserungsschritten haben. Um trotzdem zeitlich ein Begrenzung zu haben ist es beiden Methonden nicht erlaubt nach 4 Stunden Laufzeit das Netz nochmal zu vergrössern.

Um für die Netze eine Beschränkung zu haben sind maximal die Anzahl Parameter/Flops erlaubt, die das breite Baseline Netz hat.

MorphNet wird in Kapitel \ref{sec:morphexperimente} evaluiert. In Kapitel \ref{sec:ptexperimente} wird PruneTrain so evaluiert, wie es in der vorgefertigten Implementierung \footnote{\url{https://bitbucket.org/lph\_tools/prunetrain/src/master/}} geschrieben wurde. In Kapitel \ref{sec:ptnew} wird die Erhöhung der Batchgröße bei Beschneiden des Netzes evaluiert.

In Kapitel \ref{sec:net2netexperimente} wird Net2Net evaluiert.
In Kapitel \ref{sec:ptpnet2net} wird die Vorgehensweise zum Kombinieren von PruneTrain und Net2Net beschrieben sowie die Kombination evaluiert.

Der Vergleich der beiden Methoden wird in Kapitel \ref{sec:vergleich} vollzogen.
\todo[inline]{nur wenn noch Zeit übrig: In Kapitel 8 werden die Ergebnisse aus 6.3 auf einem anderen Datensatz verifiziert. auch wenn noch Zeit: Prüfe wie sich die additiven Verfahren auf 6.3 auswirken Zuletzt werden noch additive Verfahren vorgetellt, welche die Trainingszeit zusätzlich minimieren können. Eine dieser Verfahren, welches in Kapitel evaluiert wird, spart Zeit durch die Verwendung von gemischt präzisen Zahlenformaten.
Ein weiteres additives Verfahren in Kapitel  überprüft in wiefern mit Hilfe einer adaptiven Anpassung der Lernrate die Batchgröße sinvoll so angepasst werden kann, dass die ganze GPU genutzt werden kann.}








\chapter{Durchführung}

\begin{itemize}
 \item Baseline Training ab der ersten Epoche.
 \item Prune währenddem Training
 \item Training bis zu dem Zeitpunkt wo durch ein weitere Epoche nichts besser wird
 \item Überprüfe wieviel in den letzten Epochen gepruned wurde um zu entscheiden ob das Netz weiter odertiefersein soll
\end{itemize}



% Anhang ---------------------------------------------------------------
%
\cleardoublepage
\appendix

%========================================================================================
% TU Dortmund, Computer Science VII
%========================================================================================
\chapter{d}


%
\listoffigures
\addcontentsline{toc}{chapter}{Abbildungsverzeichnis}
\cleardoublepage

%
\listofalgorithms
\addcontentsline{toc}{chapter}{Algorithmenverzeichnis}
\cleardoublepage

%
\renewcommand{\lstlistlistingname}{Quellcodeverzeichnis}
\lstlistoflistings
\addcontentsline{toc}{chapter}{Quellcodeverzeichnis}
\cleardoublepage

%
\addcontentsline{toc}{chapter}{Literaturverzeichnis}
\bibliographystyle{alpha}
\bibliography{Literature}

% ----------------------------------------------------------------------

\end{document}
