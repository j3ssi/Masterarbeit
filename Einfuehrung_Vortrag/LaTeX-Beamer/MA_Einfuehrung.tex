%========================================================================================
% Latex-Beamer-Template
% TU Dortmund, Informatik Lehrstuhl VII
%========================================================================================
%\documentclass[10pt]{beamer}
\documentclass[10pt]{beamer}

\usetheme{tufi}
\usepackage{wasysym}
\usepackage{ucs}
\usepackage[ngerman]{babel}
\usepackage[utf8]{inputenc}
\usepackage{amsmath,amsfonts,amssymb}
\usepackage{graphicx}
\usepackage[T1]{fontenc}
\usepackage{verbatim}
\usepackage[babel,german=quotes]{csquotes}
\usepackage{array}
\usepackage{multirow}
\usepackage{rotating}
\usepackage{pgfpages}
\newcommand\tabrotate[1]{\begin{turn}{90}\rlap{#1}\end{turn}}

\newcommand\MyBox[2]{
  \fbox{\lower0.75cm
    \vbox to 1.7cm{\vfil
      \hbox to 1.7cm{\hfil\parbox{1.4cm}{#1\\#2}\hfil}
      \vfil}%
  }%
}

%========================================================================================
% Hier Vortragstitel, Autor und Vortragsdatum eintragen
\pdfinfo
{
  /Title       (Effizientes Training eines CNNs im Kontext bla bla blubb)
  /Creator     (TeX)
  /Author      (Jessica Bühler)
}


\title{Masterarbeit -- Effizientes Training eines CNNs im Kontext bla bla blubb}
\author{Jessica Bühler}
\date{\today}
%========================================================================================

\begin{document}

\frame{\titlepage}

\AtBeginSection[]
{
  \frame<handout:0>
  {
    \frametitle{Übersicht}
    \tableofcontents[currentsection,hideallsubsections]
  }
}

\AtBeginSubsection[]
{
  \frame<handout:0>
  {
    \frametitle{Übersicht}
    \tableofcontents[sectionstyle=show/hide,subsectionstyle=show/shaded/hide]
  }
}

\newcommand<>{\highlighton}[1]{%
  \alt#2{\structure{#1}}{{#1}}
}

\newcommand{\icon}[1]{\pgfimage[height=1em]{#1}}


%=Inhalt=================================================================================
\section{Einführung}

\begin{frame}{Fragestellung}
Ein Traininsdurchlauf von CNNs kann sehr zeitaufwendig sein. Wird dieser Prozess dann mehrfach durchlaufen, in dem verschiedene Hyperparameter / Startwerte probiert werden kann dieser Prozess sehr schnell zeitlich explodieren. Wie lässt sich das Finden der besten Architektur in Bezug auf die Accuracy effizienter gestalten?
\end{frame}

\section{Kernel rescaling}


\section{dvolver}







\end{document}
