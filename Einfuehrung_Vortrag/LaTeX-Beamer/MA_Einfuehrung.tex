%========================================================================================
% Latex-Beamer-Template
% TU Dortmund, Informatik Lehrstuhl VII
%========================================================================================
\documentclass[10pt]{beamer}

\usetheme{tufi}
\usepackage{wasysym}
\usepackage[ngerman]{babel}
\usepackage[utf8]{inputenc}
\usepackage{amsmath,amsfonts,amssymb}
\usepackage{graphicx}
\usepackage[T1]{fontenc}
\usepackage{verbatim}
\usepackage[babel,german=quotes]{csquotes}
\usepackage{array}
\usepackage{multirow}
\usepackage{rotating}
\usepackage{pgfpages}
\usepackage[backend=biber]{biblatex}
\bibliography{Literatur.bib}

\newcommand\tabrotate[1]{\begin{turn}{90}\rlap{#1}\end{turn}}
\newcommand\MyBox[2]{
  \fbox{\lower0.75cm
    \vbox to 1.7cm{\vfil
      \hbox to 1.7cm{\hfil\parbox{1.4cm}{#1\\#2}\hfil}
      \vfil}%
  }%
}

%========================================================================================
% Hier Vortragstitel, Autor und Vortragsdatum eintragen
\pdfinfo
{
  /Title       (Zeit-Effizientes Training von Convolutional Neural Networks)
  /Creator     (TeX)
  /Author      (Jessica Bühler)
}


\title{Masterarbeit -- Zeit-Effizientes Training von Convolutional Neural Networks}
\author{Jessica Bühler}
\date{\today}
%========================================================================================

\begin{document}

\frame{\titlepage}

\AtBeginSection[]
{
  \frame<handout:0>[allowframebreaks]
  {
    \frametitle{Übersicht}
    \tableofcontents[currentsection,hideallsubsections]
  }
}

\AtBeginSubsection[]
{
  \frame<handout:0>[allowframebreaks]
  {
    \frametitle{Übersicht}
    \tableofcontents[sectionstyle=show/hide,subsectionstyle=show/shaded/hide]
  }
}

\newcommand<>{\highlighton}[1]{%
  \alt#2{\structure{#1}}{{#1}}
}

\newcommand{\icon}[1]{\pgfimage[height=1em]{#1}}


%=Inhalt=================================================================================
\section{PruneTrain}
\begin{frame}{Pruning in der Vergangenheit}
Bisher:
\begin{itemize}
 \item Pruning, um vortrainierte Netze zu verkleinern
 \item Mit Hilfe von iterativen und Pruning Zyklen dünne Subnetze (Winning Tickets) finden
 \item Manche der Subnetze können über Datansets und Optimierer hin weg gerneralsieren
\end{itemize}
Aber: bisher keine effizienten Algorithmen um dies zu finden. Es benötigt in der Regel mehr Zeit als das Netz direkt zu trainieren.
\end{frame}


\begin{frame}{Fragestellung}
 Daraus resultiert eine fundamentale Fragestellung:
 Lassen sich stark verdünnte trainierbare Subnetze bereits zur Initialisierung finden? 
 Ohne Training und vielleicht auch ohne die Datensätze zu betrachten?
\end{frame}


\begin{frame}{Kollaps ganzer Schichten}

 
\end{frame}


\end{document}
